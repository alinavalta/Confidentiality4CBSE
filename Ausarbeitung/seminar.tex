%% LaTeX2e class for seminar theses
%% seminar.tex
%% 
%% Karlsruhe Institute of Technology
%% Institute for Program Structures and Data Organization
%% Chair for Software Design and Quality (SDQ)
%%
%% Dr.-Ing. Erik Burger
%% burger@kit.edu
%%
%% Version 1.0.4, 2021-06-21

%% Available page modes: oneside, twoside
%% Available languages: english, ngerman
%% Available modes: draft, final (see README)
\documentclass[twoside, ngerman]{sdqseminar}

%% ---------------------------------
%% | Information about the thesis  |
%% ---------------------------------

%% Name of the author
\author{Alina Valta}

%% Title (and possibly subtitle) of the thesis
\title{Refaktorisierung einer Architekturanalyse für Vertraulichkeit}

%% Type of the thesis 
% \thesistype{Seminar Thesis}

%% Change the institute here, ``KASTEL'' is default
% \myinstitute{Institute for \dots}

%% The advisors are PhD Students or Postdocs
\advisor{M.Sc. Frederik Reiche}

\settitle

%% --------------------------------
%% | Settings for word separation |
%% --------------------------------

%% Describe separation hints here.
%% For more details, see 
%% http://en.wikibooks.org/wiki/LaTeX/Text_Formatting#Hyphenation
\hyphenation{
% me-ta-mo-del
}

%% --------------------------------
%% | Bibliography                 |
%% --------------------------------

%% Use biber instead of BibTeX, see README
\usepackage[citestyle=numeric,style=numeric,backend=biber]{biblatex}
\addbibresource{seminar.bib}

%% ====================================
%% ====================================
%% ||                                ||
%% || Beginning of the main document ||
%% ||                                ||
%% ====================================
%% ====================================
\begin{document}

%% Set PDF metadata
\setpdf

%% Set the title
\maketitle

%% ----------------
%% |   Abstract   |
%% ----------------
 
%% The text is included from the following files:
%% - sections/abstract

\begin{abstract}
%\input{sections/abstract.tex}
\end{abstract}

%% -----------------
%% |   Main part   |
%% -----------------


%\input{sections/introduction.tex}
%%% LaTeX2e class for seminar theses
%% sections/content.tex
%% 
%% Karlsruhe Institute of Technology
%% Institute for Program Structures and Data Organization
%% Chair for Software Design and Quality (SDQ)
%%
%% Dr.-Ing. Erik Burger
%% burger@kit.edu

\section{First Content Section}
\label{ch:FirstContentSection}

%% -------------------
%% | Example content |
%% -------------------
The content chapters of your thesis should of course be renamed. How many
chapters you need to write depends on your thesis and cannot be said in general.

Check out the examples theses in the SDQWiki:

\url{https://sdqweb.ipd.kit.edu/wiki/Form_der_Ausarbeitung_bei_Seminaren}

Of course, you can split this .tex file into several files if you prefer. 


\subsection{First Subsection}
\label{sec:FirstContentSection:FirstSubSection}

\dots

\subsection{A Subsection}
\label{sec:FirstContentSection:FirstSubSubSection}

\dots


\section{Second Content Section}
\label{ch:SecondContentSection}

\dots

\subsection{First Subsection}
\label{sec:SecondContentSection:FirstSubsection}

\dots

\subsection{Second Subsection}
\label{sec:SecondContentSection:SecondSubsection}

\dots

Add additional content sections if required by adding new .tex files in the
\code{sections/} directory and adding an appropriate 
\code{\textbackslash input} statement in \code{thesis.tex}. 
%% ---------------------
%% | / Example content |
%% ---------------------
%\input{sections/conclusion.tex}


\section{Aufgabenstellung}
Erklärung der Aufgabenstellung
\begin{itemize}
	\item Entfernen der Profils 
\end{itemize}

\section{Vertraulichkeitsanalyse}
Grobe Beschreibung wie das bisherige Projekt funktioniert 
\begin{itemize}
	\item Modellierung der Vertraulichkeit im Model + Stereotypen mit Referenzen auf PCM Komponenten
	\item Editor-Generierung
	\item PCM2Prolog Generator
	\item Weitere Schritte: Vertraulichkeitsanalyse mit Prolog (nur grob)
\end{itemize}

\section{Model}
\begin{itemize}
	\item Profile entfernt
	\item Genauere Modellierung der InformationFlows
\end{itemize}

\section{Prolog Generator}
\begin{itemize}
	\item Wie kann das andere Modell auf den gleichen Prolog Code abgebildet werden
\end{itemize}

%% --------------------
%% |   Bibliography   |
%% --------------------

%% Add entry to the table of contents for the bibliography
\printbibliography[heading=bibintoc]

\end{document}